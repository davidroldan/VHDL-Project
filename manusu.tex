\documentclass{article}

\usepackage[utf8x]{inputenc}
\usepackage[T1]{fontenc}
\usepackage{palatino}
\usepackage[spanish]{babel}
\usepackage{anysize}
\usepackage{tabu}
\usepackage{graphicx}

\usepackage{tikz}
\usetikzlibrary{trees}

\PrerenderUnicode{í}
\PrerenderUnicode{ó}

\pdfinfo{
	/Title (Teclado: manual de uso)
	/Subject (Tecnología y Organización de Computadores)
}

\title{Teclado: manual de uso}
\author{Juan Andrés Claramunt Pérez, David Roldán Santos y Rubén Rafael Rubio Cuéllar}

\begin{document}

\maketitle

\section{Ficheros y dependencias}

% Separación de las filas de la tabla
\let\sep\hline

% TODO: faltan archivos ¿sobran?

\begin{tabu}{| >{\itshape}X[2, l] | X[7, l] |}
	\multicolumn{2}{c}{\bfseries Archivos principales} \\ \sep
	archivero.vhd & Archivero (reproductor, grabador, memoria y comunicación serie) \\ \sep
	audiocod.vhd & Controlador del códec de audio \\ \sep
	display.vhd & Controlador del display de 7 segmentos \\ \sep
	estabilizador.vhd & Estabilizador de señal de 1 bit \\ \sep
	gensonido.vhd & Generador de ondas sonoras binarias \\ \sep
	grabador.vhd & Grabador \\ \sep
	pines.ucf & Pines \\ \sep
	reconocedor.vhd & Reconocedor de teclado \\ \sep
	reproductor.vhd & Reproductor \\ \sep
	segments.vhd & Conversor de binario a display de 7 segmentos \\ \sep
	tablanotas.vhd & Tabla de semiperiodos de nota \\ \sep
	tablanotassos.vhd & Tabla de semiperiodos de nota (con sostenido) \\ \sep
	teclado.vhd & Entidad principal \\ \sep
	tipos.vhd & Paquete de tipos y definiciones comunes \\ \sep
	uart\_rx.vhd & Transistor de la UART (comunicación serie) \\ \sep
	uart\_tx.vhd & Receptor de la UART (comunicación serie) \\ \sep
	vga.vhd & Controlador de pantalla \\ \sep
	vga\_barras.vhd & Componente de barras (pantalla) \\ \sep
	vga\_recButton.vhd & Señal de grabación (pantalla) \\ \sep
	vga\_teclado.vhd & Componente teclado (pantalla) \\ \sep

	% ¿Incluimos estos archivos?
	\multicolumn{2}{c}{} \\
	\multicolumn{2}{c}{{\bfseries Simulaciones} en {\itshape pruebas}} \\ \sep
	sim\_archivero.vhd & Simulación del archivero \\ \sep
	sim\_codec.vhd & Simulación del códec \\ \sep
	sim\_grab.vhd & Simulación del grabador \\ \sep
	sim\_repr.vhd & Simulación del reproductor \\ \sep

	\multicolumn{2}{c}{} \\
	\multicolumn{2}{c}{{\bfseries Conversor BMP a VHD} en {\itshape conversorBMPtoVHD}} \\ \sep
	converter.cpp & Archivo principal \\ \sep
\end{tabu}

\vspace*{1cm}

\begin{figure}[ht] \centering
\begin{tikzpicture}[level 1/.style={sibling distance=20mm}]

	\node {archivero} [style=edge from parent fork down]
		child { node {reproductor} }
		child { node {grabador} }
		child { node {uart\_rx} }
		child { node {uart\_tx} }
	;
\end{tikzpicture}
\caption{Árbol de dependencias de los archivos de la descripción {\itshape hardware} (archivero)}
\end{figure}

\begin{figure}[ht] \centering
\begin{tikzpicture}[level 1/.style={sibling distance=20mm}]

	\node {teclado} [style=edge from parent fork down]
		child {
			node {gensonido}
			child { node {tablanotas} }
			child { node {tablanotassos} }
		}
		child { node {reconocedor} }
		child {
			node {archivero*}
		}
		child {
			node {vga}
			child { node {vga\_barras} }
			child { node {vga\_\ldots} }
			child { node {vga\_teclado} }
		}
		child { node {audiocod} }
		child {
			node {display}
			child { node {segments} }
		}
		child { node {pines.ucf} }
	;
\end{tikzpicture}
\caption{Árbol de dependencias de los archivos de la descripción {\itshape hardware}}
\end{figure}

\begin{tabu}{| >{\itshape}X[3, l] | X[5, l] |}
	\multicolumn{2}{c}{{\bfseries Manipulador de archivos FLAN} en {\itshape mflan}} \\ \sep
	Léeme.txt & Instrucciones de compilación \\ \sep
	Makefile & Archivo de construcción (GNU Makefile) \\ \sep
	Makefile.mak & Archivo de construcción (Microsoft NMake) \\ \sep
	lylector.cpp & Lector de archivos de partitura \\ \sep
	lylector.h & \\ \sep
	main.cpp & Archivo principal \\ \sep
	mflan.cpp & Clase principal \\ \sep
	mflan.h & \\ \sep
	nota.h & Clase abstracta Nota. \\ \sep
	notaFPGA.cpp & Nota del formato FLAN \\ \sep
	notaFPGA.h & \\ \sep
	ondaseno.cpp & Onda senoidal (con PortAudio) \\ \sep
	ondaseno.h & \\ \sep
	operacion.h & Clase abstracta Operación \\ \sep
	ops/op\_convertir.cpp & Operación de conversión \\ \sep
	ops/op\_convertir.h & \\ \sep
	ops/op\_escalar.cpp & Operación de cambio de escala \\ \sep
	ops/op\_escalar.h & \\ \sep
	ops/op\_leer.cpp & Operación de lectura y comprobación \\ \sep
	ops/op\_leer.h & \\ \sep
	ops/op\_reproducir.cpp & Operación de reproducción \\ \sep
	ops/op\_reproducir.h & \\ \sep
	tamborilero.txt & Archivo de partitura de ejemplo \\ \sep

	\multicolumn{2}{c}{} \\
	\multicolumn{2}{c}{{\bfseries Cargador/descargador de archivos} en {\itshape eurotas}} \\ \sep
	Léeme.txt & Intrucciones de compilación y uso \\ \sep
	toc/Main.java & Clase principal \\ \sep
	toc/OyenteTarea.java & Interfaz de oyente de tarea \\ \sep
	toc/comm/Cargador.java & Cargador de archivos en la FPGA \\ \sep
	toc/comm/Comunicador.java & Clase abstracta: comunicador con la FPGA \\ \sep
	toc/comm/Descargador.java & Descargador de archivos desde la FPGA \\ \sep
	toc/comm/ErrorComunicacion.java & Excepción de error en la comunicación \\ \sep
	toc/comm/OyenteSerieDebug.java & Oyente de eventos de la comunicación \\ \sep
	toc/comm/Saludador.java & Saludador de la FPGA \\ \sep
	toc/comm/package-info.java & Información del paquete {\itshape toc.comm} \\ \sep
	toc/gui/MonitorProgreso.java & Diálogo monitor del progreso \\ \sep
	toc/gui/PanelArchivo.java & Panel de selección de archivos \\ \sep
	toc/gui/PanelArchivoListener.java & Oyente del selector de archivos s\\ \sep
	toc/gui/VentanaPpal.java & Ventana principal \\ \sep
	toc/gui/package-info.java & Información del paquete {\itshape toc.gui} \\ \sep
	toc/package-info.java & Información del paquete {\itshape toc} \\ \sep
\end{tabu}

\section{Pines}

\begin{tabu}{|>{\bfseries}X[1, l] | >{\tt}X[2, l] | >{\itshape}X[1, l] |X[10, l] |}
	\sep T9 & reloj & in & Señal de reloj de la FPGA \\ \sep
	J4 & reset & in & Interruptor de reinicio \\ \sep
	D15 & onda & out & Señal de onda \\ \sep

	\multicolumn{4}{c}{\bfseries Teclado PS/2} \\ \sep	
	B16 & PS2CLK & in & Reloj del teclado PS/2 \\ \sep
	E13 & PS2DATA & in & Bit de datos del teclado PS/2 \\ \sep
	
	\multicolumn{4}{c}{\bfseries Códec de audio} \\ \sep
	P11 & au\_mclk & out & Reloj principal \\ \sep
	R12 & au\_lrck & out & Selector de canal \\ \sep
	T12 & au\_bclk & out & Reloj de la entrada en serie del códec \\ \sep
	M10 & au\_sdti & out & Entrada de datos \\ \sep

	\multicolumn{4}{c}{\bfseries Pantalla} \\ \sep
	B7 & hsyncb & in & Bit de sincronía horizontal \\ \sep
	D8 & vsyncb & inout? & Bit se sincronía vertical \\ \sep

	\multicolumn{4}{c}{\bfseries LED indicadores} \\ \sep
	L5 & led\_repr & out & LED que indica si se está reproduciendo \\ \sep
	N2 & led\_grab & out & LED que indica si se está grabando \\ \sep
	M3 & led\_trans & out & Indica si se están transmitiendo datos por el puerto serie \\ \sep

	\multicolumn{4}{c}{\bfseries Puerto serie} \\ \sep
	G5 & rs232\_rd & in & Entrada de datos \\ \sep
	J2 & rs232\_td & out & Salida de datos \\ \sep
\end{tabu}

	\bigskip Otros conjuntos de pines son:

\medskip \begin{tabu}{| X[5, l] | >{\tt}X[1,l] | X[1, l] | X[5, l] |}
	\sep
	Señal RGB para la pantalla & rgb & 0..8 & (D5, E7, C9, C3, A5, A8, B1, D6, C8) \\ \sep
	Display de 7 segmentos izquierdo & dspiz & 0..6 & (H14, M4, P1, N3, M15, H13, G16) \\ \sep
	Display de 7 segmentos derecho & dspdr & 0..6 & (E2, E1, F3, F2, G4, G3, G1) \\ \sep
\end{tabu} 

\section{Inicialización}

	Antes de comenzar, es necesario conectar un teclado al puerto PS2 de la FPGA. Si se dispone de un zumbador o semejante (como por ejemplo el de la placa de pruebas del laboratorio) se puede utilizar como altavoz conectándolo al pin {\color{blue} D15}. Si por el contrario, se pretende escuchar utilizando auriculares u otros dispositivos que cuenten con conector compatible se podrán conectar estos al puerto a tal efecto de la FPGA, aunque su funcionamiento puede depender de la FPGA utilizada. Para ello la disposición de los {\itshape jumper} JP2-JP5 ha de ser la que indica la figura \ref{fig:jump}.

\begin{figure}[ht] \centering
	\includegraphics[scale=.3]{poscodec.png}

	\caption{Posición de los {\itshape jumper} para eschuchar con auriculares}
	\label{fig:jump}
\end{figure}


	\medskip Para poder contemplar los contenidos visuales será necesario conectar una pantalla a la FPGA mediante el conector VGA.

	\medskip Para permitir la transferencia de archivos de audio entre el ordenador y la FPGA, se ha de conectar el cable serie al puerto RS232 de la placa. La configuración de los {\itshape jumper} del puerto serie ha de ser idéntica a la de la figura \ref{fig:rs232} (conexión null módem).

% ¿La conexión es ésta?
\begin{figure}[ht] \centering
	\includegraphics[scale=.25]{posserie.png}

	\caption{Posición de los {\itshape jumper} para la transferencia de archivos}
	\label{fig:rs232}
\end{figure}

\section{Funcionamiento}

	El teclado funciona en diferentes modos:

	\begin{enumerate}
		\item {\itshape Modo teclado}: por defecto el sistema funciona como un piano cuyo teclado es el teclado conectado a la FPGA. Las diferentes notas (dos octavas y media simultáneamente) se distribuyen de acuerdo al esquema de la figura \ref{fig:teclado}. Se puede variar la octava con las flechas verticales del teclado numérico. El número de octava se muestra en el display de 7 segmentos al pulsar un nota natural. Al pulsar las teclas se produce sonido y simultáneamente se muestran diversas animaciones en la pantalla.

\begin{figure}[ht] \centering
	\includegraphics[scale=1.8]{keyboard.pdf}

	\caption{Disposición del teclado}
	\label{fig:teclado}
\end{figure}

	\item {\itshape Modo reproducción y grabación}: la fila central del teclado numérico permite activar o desactivar la reproducción o la grabación de pistas almacenadas en los bloques de memoria de la FPGA. Se cuenta con 20 bloques activos que permiten almacenar cada uno hasta 12 horas de grabación (en silencio) con una sensibilidad de 167,77 ms. Para cambiar de bloque de memoria se dispone de las teclas {\itshape AvPág} y {\itshape RePág} del teclado numérico. El display de 7 segmentos mostrará en reposo el bloque de memoria seleccionado.

		Durante la reproducción sonará el contenido del bloque de memoria seleccionado mostrándose en la pantalla como si se tocase a mano. Las pulsaciones en la parte sonora del teclado serán ignoradas.

		En la grabación se registrará aquello que se esté tocando con el teclado.

	\item {\itshape Modo transmisión}: en todo momento se podrán transferir archivos de audio desde y hasta los bloques de memoria de la FPGA, siempre que el bloque requerido no esté ocupado por por una grabación o reproducción. Se dispone de un programa para el ordenador que permite escoger archivos y enviarlos o generarlos a partir de los datos recibidos.

	\end{enumerate}
\section{¿Qué espero ver?}

\end{document}
