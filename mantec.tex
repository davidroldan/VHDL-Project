\documentclass{article}

\usepackage[utf8x]{inputenc}
\usepackage[T1]{fontenc}
\usepackage{palatino}
\usepackage[spanish]{babel}
\usepackage{anysize}
\usepackage{tabu}
\usepackage{graphicx}

\PrerenderUnicode{é}
\PrerenderUnicode{í}
\PrerenderUnicode{ó}

\pdfinfo{
	/Title (Teclado: manual técnico)
	/Subject (Tecnología y Organización de Computadores)
}

\title{Teclado: manual técnico}
\author{Juan Andrés Claramunt Pérez, David Roldán Santos y Rubén Rafael Rubio Cuéllar}

\begin{document}

\maketitle

\section{Ruta de datos}

	\begin{figure}[ht]\centering
		\includegraphics[scale=.62]{esquema.pdf}
		\caption{Esquema de módulos e interconexión}
		\label{fig:esquema}
	\end{figure}

	En la figura \ref{fig:esquema} las líneas gruesas verdes indican el paso de los ``parámetros musicales'': nota (3 bits), octava (3 bits) y sostenido (3 bits).

\section{Controladores}

\begin{enumerate}
	\item {\itshape Controlador de teclado}: 
	\item {\itshape Controlador de pantalla}:
	\item {\itshape Controlador de comunicación serie}:
	\item {\itshape Controlador del códec de audio}:
\end{enumerate}

\end{document}
