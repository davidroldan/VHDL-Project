\documentclass{article}

\usepackage[utf8x]{inputenc}
\usepackage[T1]{fontenc}
\usepackage{palatino}
\usepackage[spanish, es-noquoting]{babel}
\usepackage{anysize}
\usepackage{tabu}
\usepackage{graphicx}
\usepackage{tikz}

\usetikzlibrary{positioning}

\tikzset{modulo/.style={fill=yellow!80!black}, lineamus/.style={draw=green!80!black, thick}}

\PrerenderUnicode{é}
\PrerenderUnicode{í}
\PrerenderUnicode{ó}

\pdfinfo{
	/Title (Teclado: manual técnico)
	/Subject (Tecnología y Organización de Computadores)
}

\hyphenpenalty=5000
\tolerance=1000

\title{Teclado: manual técnico}
\author{Juan Andrés Claramunt Pérez, David Roldán Santos y Rubén Rafael Rubio Cuéllar}

\begin{document}

\maketitle

\section{Ruta de datos}

	\begin{figure}[ht]\centering
		\includegraphics[scale=.62]{esquema.pdf}
		\caption{Esquema de módulos e interconexión}
		\label{fig:esquema}
	\end{figure}

\begin{center}
\begin{tikzpicture}[node distance=5mm and 5mm]
	% Módulos de diseño del proyecto
	\node (UC) [modulo] {unidad central};
	\node (Gen) [above left= -1mm and 1cm of UC.north west, modulo] {gensonido};
	\node (Codec) [below left= 1mm and 1cm of UC.south west, modulo] {códec de audio};
	\node (Vga) [below=1 cm of UC, modulo] {vga};
	\node (VgaBarra) [below left=of Vga, modulo] {vga\_barra};
	\node (VgaTeclado) [below right=of Vga, modulo] {vga\_teclado};
	\node (Recon) [above=1cm of UC, modulo] {reconocedor}; 
	\node (Arch) [right=1.5 cm of UC, modulo] {archivero};
	\node (Mem) [above =of Arch, modulo] {(memoria)};
	\node (UART) [below=of Arch, modulo] {UART};
	\node (Grab) [above right= 0mm and 4mm of Arch.north east, modulo] {grabador};
	\node (Repr) [below right= 0mm and 4mm of Arch.south east, modulo] {reproductor};

	% Conexiones entre los módulos
	\draw [<-] (UC.180) -- (Gen);
	\draw [->, lineamus] (UC.north west) -- (Gen.5);
	\draw [<->, lineamus] (UC) -- (Arch);
	\draw (Arch) -- (Grab);
	\draw (Arch) -- (Repr);
	\draw (Arch) -- (Mem);
	\draw (Arch) -- (UART);
	\draw [->, lineamus] (UC) -- (Vga);
	\draw (Vga) -- (VgaBarra);
	\draw (Vga) -- (VgaTeclado);
	\draw [<->] (UC.south west) -- (Codec.0);
	\draw [->, lineamus]  (Recon.-40) -- (UC.40);
	\draw [->] (Recon.-140)-- (UC.140);
\end{tikzpicture}
\end{center}


\begin{figure}[ht]\centering
	\begin{tikzpicture}
		% Módulos interconectados
		\node (UC) [modulo, minimum width=10cm] {unidad central};
		\node (Recon) [above=1cm of UC, modulo, minimum width=6cm] {reconocedor}; 

		% Conexiones
		\draw [->] (UC.174) -- node[left]{\small reloj/reset} (Recon.-174); % reloj/reset
		\draw [->, lineamus] (Recon.-170) -- (UC.170);
		\draw [->] (Recon.-7) -- (UC.7);		% botón octava+
		\draw [->] (Recon.-8) -- (UC.8);		% botón octava-
		\draw [->] (Recon.-10) -- (UC.10);	% botón parada
		\draw [->] (Recon.-12) -- (UC.12);	% botón reproducción
		\draw [->] (Recon.-15) -- node[left]{\small activadores} (UC.15); % botón grabación

		\draw [->, thick] (Recon.-6) -- node[right] {\small octava base} (UC.6);
	\end{tikzpicture}

	\caption{Conexiones del reconocedor}
	\label{fig:reconocedor}
\end{figure}


	En la figura \ref{fig:esquema} las líneas gruesas verdes indican el paso de los ``parámetros musicales'': nota (3 bits), octava (3 bits) y sostenido (3 bits). Este es el dato principal del diseño. Se origina en el {\itshape reconocedor} como resultado de la interacción del usuario o en el {\itshape archivero} por la reproducción de un archivo grabado. La unidad central selecciona la fuente activa en cada momento. Su destino es el  generador de sonidos ({\itshape gensonido}) que produce una onda cuadrada; también llega al controlador de pantalla que muestra una recreación del teclado y una animación, y al archivero para la grabación.

\begin{figure}\centering
	\begin{tikzpicture}
		% Módulos interconectados
		\node (UC) [modulo, minimum width=10cm] {unidad central\strut};
		\node (Gen) [above=1cm of UC, modulo, minimum width=6cm] {gensonido\strut};
		\node (Tab) [above left=2em and -8em of Gen, modulo] {tablanotas};
		\node (TabS) [above right=2em and -8em of Gen, modulo] {tablanotassos};

		% Conexiones
		\draw [->] (UC.172) -- node[left]{\small reloj/reset} (Gen.-172); % reloj/reset
		\draw [->, lineamus] (UC.150) -- (Gen.-150);
		\draw [->] (Gen.-30) -- node[right]{\small onda} (UC.30);

		% Conexiones con las ROM
		\draw [->] (Gen.168) -- node[left]{\small nota} (Tab.-150);
		\draw [<-] (Gen.153) -- (Tab.-29);
		\draw [->] (Gen.30) -- (TabS.-162);
		\draw [<-] (Gen.9) -- node[right]{\small semiperiodo} (TabS.-19);
	\end{tikzpicture}

	\caption{Conexiones del generador de sonidos}
	\label{fig:gensonido}
\end{figure}

	\medskip La señal sonora generada por {\itshape gensonido} se dirige a un pin de salida de la FPGA para ser conectada a un zumbador o semejante, pero también se conecta al códec de audio que produce una señal analógica por el puerto de salida jack de 3,5mm del dispositivo.

	\medskip La memoria se gestiona en el módulo archivero. Se organiza en 20 bloques RAM de doble puerto y anchura 16 bits compartidos por el grabador, el reproductor y el transmisor. El reproductor lee desde el puerto A, el grabador escribe en el puerto B y el transmisor realiza sendas operaciones en los puertos cruzados. La reproducción y la grabación no pueden darse simultáneamente\footnote{Técnicamente sería posible en memorias diferentes con ligeras modificaciones.}, sin embargo la carga o descarga de datos por el transmisor pueden acontecer concurrentemente con aquellas si involucran a bloques distintos. El control de la compartición se maneja en el módulo.

\section{Controladores}

\begin{enumerate}
	\item {\itshape Controlador de teclado}: maneja la entrada por teclado. Las pulsaciones permiten tocar las notas musicales y activar ciertos comandos comandos de la máquina. El controlador recibe datos a través de la línea \verb|PS2DATA| con reloj \verb|PS2CLK| en serie desde el teclado. En base a ello establece valores acordes para los parámetros musicales o activa señales que desencadenarán acciones en otros componentes externos.
	\item {\itshape Controlador de pantalla}:
	\item {\itshape Controlador de comunicación serie}: maneja la comunicación bidireccional con el ordenador mediante el puerto serie. El protocolo de comunicación permite enviar y recibir archivos de audio para ser cargados en la memoria.
	\item {\itshape Controlador del códec de audio}: este controlador establece las señales de reloj y convierte una onda cuadrada en un valor estéreo de 20 bits apto para el códec de audio {\itshape AK4551} de la FPGA.
\end{enumerate}

\end{document}
