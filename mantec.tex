\documentclass{article}

\usepackage[utf8x]{inputenc}
\usepackage[T1]{fontenc}
\usepackage{palatino}
\usepackage[spanish, es-noquoting]{babel}
\usepackage{anysize}
\usepackage{tabu}
\usepackage{graphicx}
\usepackage{tikz}

\usetikzlibrary{positioning}

\PrerenderUnicode{é}
\PrerenderUnicode{í}
\PrerenderUnicode{ó}

\pdfinfo{
	/Title (Teclado: manual técnico)
	/Subject (Tecnología y Organización de Computadores)
}

\title{Teclado: manual técnico}
\author{Juan Andrés Claramunt Pérez, David Roldán Santos y Rubén Rafael Rubio Cuéllar}

\begin{document}

\maketitle

\section{Ruta de datos}

	\begin{figure}[ht]\centering
		\includegraphics[scale=.62]{esquema.pdf}
		\caption{Esquema de módulos e interconexión}
		\label{fig:esquema}
	\end{figure}

\begin{center}
\begin{tikzpicture}[modulo/.style={fill=yellow!80!black}, lineamus/.style={draw=green!80!black, thick}, node distance=5mm and 5mm]
	\node (UC) [modulo] {unidad central};
	\node (Gen) [above left= -1mm and 1cm of UC.north west, modulo] {gensonido};
	\node (Codec) [below left= -1mm and 1cm of UC.south west, modulo] {códec de audio};
	\node (Vga) [below=1 cm of UC, modulo] {vga};
	\node (VgaBarra) [below left=of Vga, modulo] {vga\_barra};
	\node (VgaTeclado) [below right=of Vga, modulo] {vga\_teclado};
	\node (Recon) [above=1cm of UC, modulo] {reconocedor}; 
	\node (Arch) [right=1.5 cm of UC, modulo] {archivero};
	\node (Mem) [above =of Arch, modulo] {(memoria)};
	\node (UART) [below=of Arch, modulo] {UART};
	\node (Grab) [above right= 0mm and 4mm of Arch.north east, modulo] {grabador};
	\node (Repr) [below right= 0mm and 4mm of Arch.south east, modulo] {reproductor};

	\draw [->] (UC) -- (Gen);
	\draw [<->, lineamus] (UC) -- (Arch);
	\draw (Arch) -- (Grab);
	\draw (Arch) -- (Repr);
	\draw (Arch) -- (Mem);
	\draw (Arch) -- (UART);
	\draw [->] (UC) -- (Vga);
	\draw (Vga) -- (VgaBarra);
	\draw (Vga) -- (VgaTeclado);
	\draw (UC) -- (Codec);
	\draw [->, lineamus]  (Recon) -- (UC);
	\draw [->] (Recon)-- (UC);
\end{tikzpicture}
\end{center}

	En la figura \ref{fig:esquema} las líneas gruesas verdes indican el paso de los ``parámetros musicales'': nota (3 bits), octava (3 bits) y sostenido (3 bits).

\section{Controladores}

\begin{enumerate}
	\item {\itshape Controlador de teclado}: 
	\item {\itshape Controlador de pantalla}:
	\item {\itshape Controlador de comunicación serie}:
	\item {\itshape Controlador del códec de audio}:
\end{enumerate}

\end{document}
